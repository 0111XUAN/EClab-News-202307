\documentclass{eclab-beamer}

\usepackage{soul}

\title{\sffamily 东西情报}
\subtitle{\sffamily 7月刊}

\begin{document}

\setbeamercolor{background canvas}{bg=Titlebg}
\begin{frame}
  \titlepage
\end{frame}
\setbeamercolor{background canvas}{bg=}

\begin{frame}{\sffamily Editor Board}

  \begin{itemize}

    \item \sffamily Editor-in-Chief

      \begin{itemize}

\item Qing Gu, Undergraduate, 2018
\item Hongxu Zhou, Undergraduate, 2019

      \end{itemize}

    \item \sffamily Associate Editor

      \begin{itemize}

\item Shuoan Li, Undergraduate, 2022
\item Name, Undergraduate, 2022

      \end{itemize}

    \item \sffamily Consuling Editor
      \begin{itemize}
        \item \sffamily Xia Fang, PhD, Professor
      \end{itemize}

  \end{itemize}

\end{frame}

\begin{frame}[allowframebreaks]{\sffamily Journal List}
  \begin{itemize}\centering

\item Emotion
\item Journal B
\item 2Journal B

  \end{itemize}
\end{frame}

\begin{frame}[allowframebreaks]{\sffamily 卷首语}

\begin{itemize}

\item Emotion研究更新了3篇
\begin{enumerate}
\item \href{https://doi.org/10.1037/emo0001099}{\color{blue} \ul{Parental emotion socialization: Relations with adjustment, security, and maternal depression in early adolescence.}}

\footnotesize{Waslin, Stephanie M. Kochendorfer, Logan B. Blier, Brittany Brumariu, Laura E. Kerns, Kathryn A.}

本研究探讨青少年早期父母使用的情绪社会化(ES)策略(研究1),然后检查了ES策略与青春期早期调整,亲子依恋和母亲抑郁的关系(研究2)。研究发现,父母在青少年早期使用了6种传统的ES策略(解决问题,以情绪为中心/安慰,鼓励,最小化,惩罚和痛苦),同时还使用了3种在幼儿父母研究中未发现的方法(自我调节,父母寻求信息,父母解释)
\item \href{https://doi.org/10.1037/emo0001094}{\color{blue} \ul{Initial evidence for a relation between behaviorally assessed empathic accuracy and affect sharing for people and music.}}

\footnotesize{Tabak, Benjamin A. Wallmark, Zachary Nghiem, Linh H. Alvi, Talha Sunahara, Cecile S. Lee, Junghee Cao, Jing}

本研究探讨移情过程的行为评估与音乐之间的关系,线性混合效应模型显示,对于讲述个人故事和音乐表达的人,移情准确性与情感分享之间存在正相关关系,并且当将相关的个体差异作为协变量时,结果保持不变。
\item \href{https://doi.org/example}{\color{blue} \ul{This is a title}}

\footnotesize{A B \& C D}

This is a summary
\end{enumerate}
\item Face研究更新了1篇
\begin{enumerate}
\item \href{https://doi.org/example2}{\color{blue} \ul{2This is a title}}

\footnotesize{2A B \& C D}

2This is a summary
\end{enumerate}

\end{itemize}
\end{frame}


\begin{frame}[allowframebreaks]{\color{black} \normalsize{\ul{Emotion}} \hfill
\begin{tabular}{r}
\textit{Emotion, 23(2), 450-459}\\
\href{https://doi.org/10.1037/emo0001099}{\color{blue} \footnotesize{\ul{\textit{https://doi.org/10.1037/emo0001099}}}}
\end{tabular}}

\textbf{\Large{Parental emotion socialization: Relations with adjustment, security, and maternal depression in early adolescence.}}
\vspace{1mm}
Waslin, Stephanie M. Kochendorfer, Logan B. Blier, Brittany Brumariu, Laura E. Kerns, Kathryn A.

\hspace*{\fill}


\textbf{Abstract:}
How parents approach and teach their children about emotions are key determinants of children’s healthy adjustment (Denham, 2019). Parental emotion socialization has been mostly studied in parents of young children. Our study identified emotion socialization (ES) strategies used by parents of early adolescents (Study 1) and then examined the relations of ES strategies with early adolescent adjustment, parent–child attachment, and maternal depression (Study 2). Study 1 included 171 parents of 9- to 14-year-old children who completed an open-ended questionnaire about their reactions to their children’s negative emotions, which was content coded for ES strategies. We found that parents do use the 6 traditional ES strategies (problem solving, emotion focused/comforting, encouragement, minimizing, punitive, and distress) with early adolescents, while also using 3 approaches not identified in studies of parents of younger children (self-regulation, parent seeking information, parent explaining). We also found that some ES strategies are context and gender specific. Study 2 included 218 mother and child dyads (children aged 9- to 14- years). Mothers completed the Revised Coping with Children’s Negative Emotions Scale, adapted to include items assessing the 3 new strategies, and measures of child adjustment, attachment, and maternal depression. The ES strategies loaded on 3 factors: Collaborative Coping, Negative Reactions to Child’s Distress, and Low Expectation for Child’s Self-Regulation. Negative Reactions to Child’s Distress showed associations with children’s internalizing, externalizing, and prosocial behavior, and child attachment, while Collaborative Coping was related to prosocial behavior. Our results point to the importance of investigating additional ES strategies in early adolescence.

\textbf{Keywords:} emotion socialization, ES strategies , early adolescents
\end{frame}
\begin{frame}[allowframebreaks]{\color{black} \normalsize{\ul{Emotion}} \hfill
\begin{tabular}{r}
\textit{Emotion, 23(2), 437-449}\\
\href{https://doi.org/10.1037/emo0001094}{\color{blue} \footnotesize{\ul{\textit{https://doi.org/10.1037/emo0001094}}}}
\end{tabular}}

\textbf{\Large{Initial evidence for a relation between behaviorally assessed empathic accuracy and affect sharing for people and music.}}
\vspace{1mm}
Tabak, Benjamin A. Wallmark, Zachary Nghiem, Linh H. Alvi, Talha Sunahara, Cecile S. Lee, Junghee Cao, Jing

\hspace*{\fill}


\textbf{Abstract:}
Are people who are better able to understand or feel the emotions of others also better at understanding or feeling emotions conveyed through music? Although evolutionary theories have proposed that both empathy and music help to foster social connection, few studies to date have examined the relation between behavioral assessments of empathic processes for people and music. We examined this question using 2 independent samples: a laboratory sample of undergraduates (n = 236) and a larger online direct replication with participants across the United States (n = 596). Across both samples, linear mixed effects models showed positive associations between empathic accuracy and affect sharing for people telling personal stories and for musical expression, and results were maintained when including relevant individual differences as covariates. These findings provide initial evidence of a relation between behaviorally assessed empathic processes across social and musical domains. Future research is needed to build upon this evidence by investigating whether active, socially engaged music listening may have a beneficial effect on social cognition.

\textbf{Keywords:} empathy, music, behavioral assessments
\end{frame}
\begin{frame}[allowframebreaks]{\color{black} \normalsize{\ul{Emotion}} \hfill
\begin{tabular}{r}
\textit{Journal B, 20 July, 2023}\\
\href{https://doi.org/example}{\color{blue} \footnotesize{\ul{\textit{https://doi.org/example}}}}
\end{tabular}}

\textbf{\Large{This is a title}}
\vspace{1mm}
A B \& C D

\hspace*{\fill}


\textbf{Abstract:}
This is the abstract

\textbf{Keywords:} This is keywords
\end{frame}
\begin{frame}[allowframebreaks]{\color{black} \normalsize{\ul{Face}} \hfill
\begin{tabular}{r}
\textit{2Journal B, Volume 37, issue 1, 01 July, 2023}\\
\href{https://doi.org/example2}{\color{blue} \footnotesize{\ul{\textit{https://doi.org/example2}}}}
\end{tabular}}

\textbf{\Large{2This is a title}}
\vspace{1mm}
2A B \& C D

\hspace*{\fill}


\textbf{Abstract:}
2This is the abstract

\textbf{Keywords:} 2This is keywords
\end{frame}
\end{document}
